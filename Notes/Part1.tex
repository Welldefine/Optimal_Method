\documentclass{MathNote}

% --- Added common math packages (in case class file does not already load them) ---
\usepackage{amsmath, amssymb, bm}

% --- Temporary citation macro (numeric placeholders) ---
% Usage: \pcite{41,42}  (later replace with \cite{key1,key2})
\newcommand{\pcite}[1]{(#1)}

% NOTE: The MathNote class expects theorem-like environments of the form
%   \begin{definition}{Title}{label}
% NOT the optional-argument amsthm style. We revert them accordingly.
\title{\textbf{Notes: Optimization Methods | Week1\&2}}
\author{Runze Tian}
\date{\today}

\begin{document}

\maketitle

%\tableofcontents


\section{Introduction to Optimization}

\subsection{Basic Concepts and Problem Formulation}

\begin{definition}{Optimization}{def:optimization}
	Optimization is the process of selecting a best element from a set of available alternatives, with regard to some criterion. The goal is to find a set of parameters or a model that minimizes or maximizes a certain objective function.
\end{definition}

Optimization problems are central to fields like machine learning, operations research, and economics. We can formalize these problems into two primary categories.

\begin{definition}{Unconstrained Optimization}{def:unconstrained}
	An unconstrained optimization problem is formulated as:
	$$
	\min_{x \in \mathbb{R}^n} f(\bm{x})
	$$
	Here, $\bm{x} \in \mathbb{R}^n$ is the \textbf{decision variable}, and $f: \mathbb{R}^n \to \mathbb{R}$ is the \textbf{objective function}. A solution $\bm{x}^*$ such that $f(\bm{x}^*) \le f(\bm{x})$ for all $\bm{x} \in \mathbb{R}^n$ is called a global minimizer. The value $f(\bm{x}^*)$ is the optimal value.
\end{definition}

\begin{definition}{Constrained Optimization}{def:constrained}
	A constrained optimization problem is formulated as:
	\begin{align*}
	\min_{\bm{x} \in \mathbb{R}^n} \quad & f(\bm{x}) \\
	\text{subject to} \quad & c_i(\bm{x}) = 0, \quad i \in \mathcal{E} \\
	& c_i(\bm{x}) \ge 0, \quad i \in \mathcal{I}
	\end{align*}
	where $\mathcal{E}$ and $\mathcal{I}$ are index sets for the \textbf{equality constraints} and \textbf{inequality constraints}, respectively. The functions $c_i(\bm{x})$ are known as the constraint functions. Any maximization problem $\max f(\bm{x})$ can be converted to a minimization problem by considering $\min -f(\bm{x})$.
\end{definition}

\subsection{Classification of Optimization Problems}

Optimization problems can be classified based on the nature of their variables and functions.
\begin{itemize}
	\item \textbf{Continuous vs. Discrete Optimization}: If the decision variable $\bm{x}$ can take any real value (i.e., $\bm{x} \in \mathbb{R}^n$), the problem is continuous. If $\bm{x}$ is restricted to be an integer (i.e., $\bm{x} \in \mathbb{Z}^n$), the problem is discrete, often called combinatorial optimization, which is generally harder to solve.
	\item \textbf{Smooth vs. Non-smooth Optimization}: If the objective and constraint functions are continuously differentiable, the problem is smooth. Otherwise, it is a non-smooth problem.
	\item \textbf{Linear vs. Non-linear Programming}: If the objective function $f$ and all constraint functions $c_i$ are linear, the problem is a \textbf{Linear Program (LP)}. If the objective is quadratic and constraints are linear, it is a \textbf{Quadratic Program (QP)}. If any function is non-linear, it is a \textbf{Non-linear Program (NLP)}.
\end{itemize}

\subsection{Convex Sets}
The concept of convexity is fundamental to optimization theory because it allows us to make strong claims about the nature of optimal solutions.

\begin{definition}{Convex Set}{def:convex_set}
	A set $C \subseteq \mathbb{R}^n$ is said to be \textbf{convex} if for any two points $\bm{x}, \bm{y} \in C$ and any scalar $\theta \in [0, 1]$, the line segment connecting them is also in $C$. That is,
	$$
	\theta \bm{x} + (1 - \theta)\bm{y} \in C, \quad \forall \bm{x}, \bm{y} \in C, \theta \in [0, 1]
	$$
	Geometrically, a set is convex if the line segment between any two of its points lies entirely within the set.
\end{definition}

% 'proposition' is defined via \newtheorem -> use optional title + \label
\begin{proposition}[Properties of Convex Sets]\label{prop:convex_sets_properties}
	Let $C_1$ and $C_2$ be convex sets in $\mathbb{R}^n$.
	\begin{enumerate}
		\item The intersection $C_1 \cap C_2$ is a convex set.
		\item The Minkowski sum $C_1 + C_2 = \{\bm{x}_1 + \bm{x}_2 \mid \bm{x}_1 \in C_1, \bm{x}_2 \in C_2\}$ is a convex set.
	\end{enumerate}
\end{proposition}

\begin{proof}
	1. \textbf{Proof of Intersection}: Let $\bm{x}, \bm{y} \in C_1 \cap C_2$. This implies $\bm{x}, \bm{y} \in C_1$ and $\bm{x}, \bm{y} \in C_2$. Since $C_1$ is convex, for any $\theta \in [0, 1]$, $\theta \bm{x} + (1 - \theta)\bm{y} \in C_1$. Similarly, since $C_2$ is convex, $\theta \bm{x} + (1 - \theta)\bm{y} \in C_2$. Therefore, $\theta \bm{x} + (1 - \theta)\bm{y} \in C_1 \cap C_2$, proving the intersection is convex.
	
	2. \textbf{Proof of Minkowski Sum}: Let $\bm{a}, \bm{b} \in C_1+C_2$. Then by definition, $\bm{a} = \bm{x}_1+\bm{x}_2$ and $\bm{b} = \bm{y}_1+\bm{y}_2$ for some $\bm{x}_1, \bm{y}_1 \in C_1$ and $\bm{x}_2, \bm{y}_2 \in C_2$. For any $\theta \in [0,1]$, consider the point $\bm{z} = \theta \bm{a} + (1-\theta)\bm{b}$.
	$$ \bm{z} = \theta(\bm{x}_1+\bm{x}_2) + (1-\theta)(\bm{y}_1+\bm{y}_2) = \underbrace{(\theta \bm{x}_1 + (1-\theta)\bm{y}_1)}_{\in C_1} + \underbrace{(\theta \bm{x}_2 + (1-\theta)\bm{y}_2)}_{\in C_2} $$
	Since $C_1$ and $C_2$ are convex, the two terms are in $C_1$ and $C_2$ respectively. Thus, $\bm{z} \in C_1+C_2$, proving the sum is convex.
\end{proof}


\subsection{Convex Functions}

\begin{definition}{Convex Function}{def:convex_function}
	Let $C \subseteq \mathbb{R}^n$ be a convex set. A function $f: C \to \mathbb{R}$ is \textbf{convex} if for any $\bm{x}, \bm{y} \in C$ and any $\theta \in [0, 1]$, the following inequality holds:
	$$
	f(\theta \bm{x} + (1 - \theta)\bm{y}) \le \theta f(\bm{x}) + (1 - \theta) f(\bm{y})
	$$
	If the inequality is strict ($<$) for $\bm{x} \ne \bm{y}$ and $\theta \in (0,1)$, then $f$ is \textbf{strictly convex}. Geometrically, the chord connecting any two points on the function's graph lies on or above the graph itself.
\end{definition}

\begin{theorem}{First-Order Condition for Convexity}{theo:first_order_convexity}
	Let $f: C \to \mathbb{R}$ be a continuously differentiable function on an open convex set $C \subseteq \mathbb{R}^n$. Then $f$ is convex if and only if for all $\bm{x}, \bm{y} \in C$:
	$$
	f(\bm{y}) \ge f(\bm{x}) + \nabla f(\bm{x})^T (\bm{y} - \bm{x})
	$$
	This means the first-order Taylor approximation of a convex function at any point provides a global underestimator of the function.
\end{theorem}

\begin{proof}
	($\Rightarrow$) Assume $f$ is convex. By definition, for $\theta \in (0, 1]$ and $\bm{x}, \bm{y} \in C$:
	$$ f(\theta\bm{y} + (1-\theta)\bm{x}) \le \theta f(\bm{y}) + (1-\theta)f(\bm{x}) $$
	Rearranging gives:
	$$ f(\bm{x} + \theta(\bm{y}-\bm{x})) - f(\bm{x}) \le \theta(f(\bm{y}) - f(\bm{x})) $$
	$$ \frac{f(\bm{x} + \theta(\bm{y}-\bm{x})) - f(\bm{x})}{\theta} \le f(\bm{y}) - f(\bm{x}) $$
	Taking the limit as $\theta \to 0^+$, the left side becomes the definition of the directional derivative of $f$ at $\bm{x}$ in the direction $\bm{y}-\bm{x}$, which is $\nabla f(\bm{x})^T (\bm{y}-\bm{x})$. Thus, $\nabla f(\bm{x})^T (\bm{y}-\bm{x}) \le f(\bm{y}) - f(\bm{x})$.

	($\Leftarrow$) Assume $f(\bm{y}) \ge f(\bm{x}) + \nabla f(\bm{x})^T (\bm{y} - \bm{x})$ for all $\bm{x}, \bm{y} \in C$.
	Let $\bm{z} = \theta \bm{x} + (1-\theta) \bm{y}$ for $\theta \in [0,1]$. We apply the inequality twice:
	\begin{align*}
		f(\bm{x}) &\ge f(\bm{z}) + \nabla f(\bm{z})^T(\bm{x}-\bm{z}) \\
		f(\bm{y}) &\ge f(\bm{z}) + \nabla f(\bm{z})^T(\bm{y}-\bm{z})
	\end{align*}
	Multiply the first inequality by $\theta$ and the second by $(1-\theta)$ and add them:
	\begin{align*}
		\theta f(\bm{x}) + (1-\theta)f(\bm{y}) &\ge \theta f(\bm{z}) + (1-\theta)f(\bm{z}) + \nabla f(\bm{z})^T(\theta(\bm{x}-\bm{z}) + (1-\theta)(\bm{y}-\bm{z})) \\
		&\ge f(\bm{z}) + \nabla f(\bm{z})^T(\theta\bm{x} + (1-\theta)\bm{y} - \bm{z}) \\
		&\ge f(\bm{z}) + \nabla f(\bm{z})^T(\bm{z} - \bm{z}) = f(\bm{z})
	\end{align*}
	This shows $f(\theta \bm{x} + (1 - \theta)\bm{y}) \le \theta f(\bm{x}) + (1 - \theta) f(\bm{y})$, so $f$ is convex.
\end{proof}


\begin{theorem}{Second-Order Condition for Convexity}{theo:second_order_convexity}
	Let $f: C \to \mathbb{R}$ be a twice continuously differentiable function on an open convex set $C \subseteq \mathbb{R}^n$.
	\begin{enumerate}
		\item $f$ is convex if and only if its Hessian matrix $\nabla^2 f(\bm{x})$ is positive semidefinite for all $\bm{x} \in C$.
		\item If $\nabla^2 f(\bm{x})$ is positive definite for all $\bm{x} \in C$, then $f$ is strictly convex.
	\end{enumerate}
	Note: The converse of the second statement is not true. For example, $f(x)=x^4$ is strictly convex, but its second derivative at $x=0$ is $f''(0)=0$, which is not positive definite.
\end{theorem}
\begin{proof} (Sketch of Part 1)
	($\Rightarrow$) Assume $f$ is convex. For any $\bm{x} \in C$ and direction $\bm{d}$, Taylor's theorem gives for small $t>0$:
	$$ f(\bm{x}+t\bm{d}) = f(\bm{x}) + t\nabla f(\bm{x})^T\bm{d} + \frac{1}{2}t^2\bm{d}^T\nabla^2 f(\bm{x})\bm{d} + o(t^2\|\bm{d}\|^2) $$
	From the first-order condition, $f(\bm{x}+t\bm{d}) \ge f(\bm{x}) + t\nabla f(\bm{x})^T\bm{d}$. Combining these gives:
	$$ \frac{1}{2}t^2\bm{d}^T\nabla^2 f(\bm{x})\bm{d} + o(t^2\|\bm{d}\|^2) \ge 0 $$
	Dividing by $t^2$ and letting $t \to 0$ yields $\bm{d}^T\nabla^2 f(\bm{x})\bm{d} \ge 0$, which is the definition of positive semidefiniteness.

	($\Leftarrow$) Assume $\nabla^2 f(\bm{x})$ is positive semidefinite. By Taylor's theorem with remainder:
	$$ f(\bm{y}) = f(\bm{x}) + \nabla f(\bm{x})^T(\bm{y}-\bm{x}) + \frac{1}{2}(\bm{y}-\bm{x})^T \nabla^2 f(\bm{z})(\bm{y}-\bm{x}) $$
	for some $\bm{z}$ on the line segment between $\bm{x}$ and $\bm{y}$. Since $\nabla^2 f(\bm{z})$ is positive semidefinite, the last term is non-negative. Therefore, $f(\bm{y}) \ge f(\bm{x}) + \nabla f(\bm{x})^T(\bm{y}-\bm{x})$, which implies $f$ is convex by Theorem \ref{theo:first_order_convexity}.
\end{proof}


\section{Fundamentals of Unconstrained Optimization}

\subsection{Optimality Conditions}

Optimality conditions are mathematical statements that characterize solutions. They are essential for verifying if a point is a solution and for designing algorithms.

\begin{definition}{Local and Global Minima}{def:minima}
	Let $f: \mathbb{R}^n \to \mathbb{R}$.
	\begin{itemize}
		\item A point $\bm{x}^*$ is a \textbf{local minimizer} if there exists an $\epsilon > 0$ such that $f(\bm{x}^*) \le f(\bm{x})$ for all $\bm{x}$ with $\|\bm{x} - \bm{x}^*\| < \epsilon$.
		\item A point $\bm{x}^*$ is a \textbf{global minimizer} if $f(\bm{x}^*) \le f(\bm{x})$ for all $\bm{x} \in \mathbb{R}^n$.
		\item If the inequalities are strict ($<$) for $\bm{x} \ne \bm{x}^*$, the minimizer is called \textbf{strict}.
	\end{itemize}
\end{definition}

For convex functions, any local minimizer is also a global minimizer.

\begin{theorem}{First-Order Necessary Condition (FONC)}{theo:fonc}
	If $\bm{x}^*$ is a local minimizer of $f$ and $f$ is continuously differentiable in an open neighborhood of $\bm{x}^*$, then the gradient at that point must be zero:
	$$
	\nabla f(\bm{x}^*) = \bm{0}
	$$
\end{theorem}
\begin{proof}
	Proof by contradiction. Assume $\nabla f(\bm{x}^*) \ne \bm{0}$. Let $\bm{d} = -\nabla f(\bm{x}^*)$. Since $f$ is differentiable, the directional derivative in direction $\bm{d}$ is $\nabla f(\bm{x}^*)^T \bm{d} = -\|\nabla f(\bm{x}^*)\|^2 < 0$. This means that for a small step $\alpha > 0$ in the direction $\bm{d}$, the function value will decrease. By Taylor's theorem:
	$$ f(\bm{x}^* + \alpha \bm{d}) = f(\bm{x}^*) + \alpha \nabla f(\bm{x}^*)^T \bm{d} + o(\alpha) = f(\bm{x}^*) - \alpha \|\nabla f(\bm{x}^*)\|^2 + o(\alpha) $$
	For a sufficiently small $\alpha > 0$, the negative linear term dominates the higher-order term, so $f(\bm{x}^* + \alpha \bm{d}) < f(\bm{x}^*)$. This contradicts the assumption that $\bm{x}^*$ is a local minimizer. Thus, we must have $\nabla f(\bm{x}^*) = \bm{0}$.
\end{proof}


\begin{theorem}{Second-Order Necessary Condition (SONC)}{theo:sonc}
	If $\bm{x}^*$ is a local minimizer of $f$ and $\nabla^2 f$ exists and is continuous in an open neighborhood of $\bm{x}^*$, then $\nabla f(\bm{x}^*) = \bm{0}$ and the Hessian matrix $\nabla^2 f(\bm{x}^*)$ is positive semidefinite.
\end{theorem}

\begin{theorem}{Second-Order Sufficient Condition (SOSC)}{theo:sosc}
	Suppose that $\nabla^2 f$ is continuous in an open neighborhood of $\bm{x}^*$. If $\nabla f(\bm{x}^*) = \bm{0}$ and the Hessian matrix $\nabla^2 f(\bm{x}^*)$ is positive definite, then $\bm{x}^*$ is a strict local minimizer of $f$.
\end{theorem}
\begin{proof}
	Since $\nabla^2 f$ is continuous and $\nabla^2 f(\bm{x}^*)$ is positive definite, there exists a radius $r>0$ such that for any $\bm{x}$ with $\|\bm{x}-\bm{x}^*\| < r$, $\nabla^2 f(\bm{x})$ is also positive definite.
	For any such $\bm{x} \ne \bm{x}^*$, let $\bm{d} = \bm{x} - \bm{x}^*$. By Taylor's theorem, there is a $\bm{z}$ on the line segment between $\bm{x}^*$ and $\bm{x}$ such that:
	$$ f(\bm{x}) = f(\bm{x}^*) + \nabla f(\bm{x}^*)^T \bm{d} + \frac{1}{2} \bm{d}^T \nabla^2 f(\bm{z}) \bm{d} $$
	Since $\nabla f(\bm{x}^*) = \bm{0}$ and $\nabla^2 f(\bm{z})$ is positive definite (as $\bm{z}$ is within the radius $r$), we have:
	$$ f(\bm{x}) = f(\bm{x}^*) + \frac{1}{2} \bm{d}^T \nabla^2 f(\bm{z}) \bm{d} > f(\bm{x}^*) $$
	The inequality is strict because $\bm{d} \ne \bm{0}$. Thus, $\bm{x}^*$ is a strict local minimizer.
\end{proof}


\subsection{Structure of Iterative Methods}

Most optimization algorithms are iterative. They generate a sequence of points $\{\bm{x}_k\}$ that ideally converge to a minimizer $\bm{x}^*$. The core idea is to move from the current point $\bm{x}_k$ to a new, better point $\bm{x}_{k+1}$.

The general structure of such a method is: $\bm{x}_{k+1} = \bm{x}_k + \alpha_k \bm{d}_k$, where:
\begin{itemize}
	\item $\bm{d}_k$ is the \textbf{search direction}. It must be a descent direction.
	\item $\alpha_k > 0$ is the \textbf{step length} (or learning rate).
\end{itemize}

\begin{definition}{Descent Direction}{def:descent}
	A direction $\bm{d}_k$ is a \textbf{descent direction} from a point $\bm{x}_k$ if for a small step, the function value decreases. For a differentiable function, this is equivalent to the condition:
	$$ \nabla f(\bm{x}_k)^T \bm{d}_k < 0 $$
	Geometrically, the descent direction must form an obtuse angle with the gradient vector.
\end{definition}

There are two main strategies for choosing $\bm{d}_k$ and $\alpha_k$:
\begin{itemize}
	\item \textbf{Line Search Methods}: First, a descent direction $\bm{d}_k$ is chosen. Second, a step length $\alpha_k$ is found that minimizes $f$ along that direction, i.e., solving $\min_{\alpha > 0} f(\bm{x}_k + \alpha \bm{d}_k)$.
	\item \textbf{Trust Region Methods}: A model function (usually quadratic) is built to approximate $f$ around $\bm{x}_k$ within a "trust region" of radius $\Delta_k$. The direction and step length are determined simultaneously by minimizing the model within this region.
\end{itemize}


\subsection{Convergence of Algorithms}

\begin{definition}{Rate of Convergence}{def:convergence_rate}
	Let $\{\bm{x}_k\}$ be a sequence that converges to $\bm{x}^*$. The convergence is said to be:
	\begin{itemize}
		\item \textbf{Linear} if there is a constant $a \in (0, 1)$ such that $\lim_{k \to \infty} \frac{\|\bm{x}_{k+1} - \bm{x}^*\|}{\|\bm{x}_k - \bm{x}^*\|} = a$.
		\item \textbf{Superlinear} if the limit is $a=0$.
		\item \textbf{Quadratic (or of order 2)} if there is a constant $a$ such that $\lim_{k \to \infty} \frac{\|\bm{x}_{k+1} - \bm{x}^*\|}{\|\bm{x}_k - \bm{x}^*\|^2} = a$.
	\end{itemize}
	Quadratic and superlinear rates are much faster than linear convergence.
\end{definition}


\section{Line Search Methods}

\subsection{Step Length and Search Criteria}
Given a descent direction $\bm{d}_k$, the one-dimensional subproblem is to find a step length $\alpha_k > 0$.

\begin{definition}{Exact Line Search}{def:exact_line_search}
	An \textbf{exact line search} finds the step length $\alpha_k$ that globally minimizes the one-dimensional function $\phi(\alpha) = f(\bm{x}_k + \alpha \bm{d}_k)$ for $\alpha > 0$. This is equivalent to finding $\alpha_k$ such that $\phi'(\alpha_k)=0$, which implies:
	$$ \nabla f(\bm{x}_k + \alpha_k \bm{d}_k)^T \bm{d}_k = 0 $$
	This means the gradient at the new point is orthogonal to the search direction. However, finding this exact minimum is often computationally expensive.
\end{definition}

In practice, an \textbf{inexact line search} is used, which finds a step length that provides a sufficient decrease in the objective function without excessive computation. Simply requiring $f(\bm{x}_{k+1}) < f(\bm{x}_k)$ is not enough to guarantee convergence to a minimizer, as the decrease might be negligible.

\subsection{The Wolfe Conditions}

The Wolfe conditions are a pair of inequalities that ensure both a sufficient decrease in the function value and that the step is not excessively short.

\begin{definition}{The Wolfe Conditions}{def:wolfe}
	For constants $0 < \rho < \sigma < 1$, a step length $\alpha_k$ satisfies the \textbf{Wolfe conditions} if the following two inequalities hold:
	\begin{enumerate}
		\item \textbf{Armijo Condition (Sufficient Decrease)}:
		$$ f(\bm{x}_k + \alpha_k \bm{d}_k) \le f(\bm{x}_k) + \rho \alpha_k \nabla f(\bm{x}_k)^T \bm{d}_k $$
		This ensures the reduction in $f$ is proportional to both the step length and the directional derivative.
		\item \textbf{Curvature Condition}:
		$$ \nabla f(\bm{x}_k + \alpha_k \bm{d}_k)^T \bm{d}_k \ge \sigma \nabla f(\bm{x}_k)^T \bm{d}_k $$
		This ensures the slope at the new point is less negative than the initial slope, preventing steps that are too short.
	\end{enumerate}
	The \textbf{Strong Wolfe Conditions} replace the curvature condition with a stricter requirement:
	$$ |\nabla f(\bm{x}_k + \alpha_k \bm{d}_k)^T \bm{d}_k| \le \sigma |\nabla f(\bm{x}_k)^T \bm{d}_k| $$
    This forces the step length to be closer to a stationary point of $\phi(\alpha)$.
\end{definition}

% 'lemma' is defined via \newtheorem -> use optional title + \label
\begin{lemma}[Existence of Step Length for Wolfe Conditions]\label{lemma:wolfe_existence}
	Let $f: \mathbb{R}^n \to \mathbb{R}$ be a continuously differentiable function. If $\bm{d}_k$ is a descent direction at $\bm{x}_k$ and $f$ is bounded below along the ray $\{\bm{x}_k + \alpha \bm{d}_k \mid \alpha > 0\}$, then there exist step lengths $\alpha > 0$ that satisfy the Wolfe conditions (for any $0 < \rho < \sigma < 1$).
\end{lemma}
\begin{proof}
	Let $\phi(\alpha) = f(\bm{x}_k + \alpha \bm{d}_k)$. Since $\bm{d}_k$ is a descent direction, $\phi'(0) = \nabla f(\bm{x}_k)^T \bm{d}_k < 0$. The Armijo condition is $\phi(\alpha) \le \phi(0) + \rho \alpha \phi'(0)$.
	The line $l(\alpha) = \phi(0) + \rho \alpha \phi'(0)$ has a negative slope. Since $f$ is bounded below, $\phi(\alpha)$ is also bounded below. Thus, the line $l(\alpha)$ must eventually cross the graph of $\phi(\alpha)$, meaning there is a set of acceptable $\alpha$ values for the Armijo condition. Let $\alpha_{\max} = \sup\{\alpha \mid \phi(\alpha) \le \phi(0) + \rho \alpha \phi'(0)\}$. Since $\phi$ is continuous, $\alpha_{\max}$ is well-defined and positive.
	
	Now consider the curvature condition. By the Mean Value Theorem, for any $\alpha > 0$, there exists some $\xi \in (0, \alpha)$ such that
	$$ \phi'(\xi) = \frac{\phi(\alpha) - \phi(0)}{\alpha} $$
	At $\alpha_{\max}$, we must have $\phi(\alpha_{\max}) = l(\alpha_{\max}) = \phi(0) + \rho \alpha_{\max} \phi'(0)$.
	Substituting this into the MVT expression (with $\alpha = \alpha_{\max}$), we get $\phi'(\xi) = \rho \phi'(0)$.
	Since $\rho < \sigma$ and $\phi'(0) < 0$, we have $\rho \phi'(0) > \sigma \phi'(0)$. Therefore,
	$$ \phi'(\xi) = \nabla f(\bm{x}_k + \xi \bm{d}_k)^T \bm{d}_k = \rho \phi'(0) > \sigma \phi'(0) $$
	This shows that the step length $\xi \in (0, \alpha_{\max})$ satisfies the curvature condition. It also satisfies the Armijo condition since $\xi < \alpha_{\max}$ and $\phi'(\xi) < 0$, implying $\phi$ is still decreasing. Thus, a valid step length exists.
\end{proof}

\subsection{Convergence of Line Search Methods}

The Wolfe conditions are crucial for proving global convergence of line search methods. The following theorem, often attributed to Zoutendijk, is a cornerstone result.

\begin{theorem}{Zoutendijk's Theorem}{theo:zoutendijk}
	Consider an iterative algorithm $\bm{x}_{k+1} = \bm{x}_k + \alpha_k \bm{d}_k$, where $\bm{d}_k$ is a descent direction and $\alpha_k$ satisfies the Wolfe conditions. Suppose $f$ is bounded below in $\mathbb{R}^n$ and is continuously differentiable in an open set containing the level set $\{\bm{x} \mid f(\bm{x}) \le f(\bm{x}_0)\}$, and that the gradient $\nabla f$ is Lipschitz continuous on this set. Then
	$$
	\sum_{k \ge 0} \cos^2 \theta_k \|\nabla f(\bm{x}_k)\|^2 < \infty
	$$
	where $\theta_k$ is the angle between the search direction $\bm{d}_k$ and the negative gradient $-\nabla f(\bm{x}_k)$.
\end{theorem}
\begin{proof}
	From the second Wolfe condition:
	$$ (\nabla f_{k+1} - \nabla f_k)^T \bm{d}_k \ge (\sigma - 1) \nabla f_k^T \bm{d}_k $$
	By the Mean Value Theorem, $(\nabla f_{k+1} - \nabla f_k) = \int_0^1 \nabla^2 f(\bm{x}_k+t\alpha_k \bm{d}_k) \alpha_k \bm{d}_k dt$.
	Using Lipschitz continuity of the gradient, $\|\nabla f_{k+1} - \nabla f_k\| \le L \|\bm{x}_{k+1} - \bm{x}_k\| = L \alpha_k \|\bm{d}_k\|$.
	So, $L \alpha_k \|\bm{d}_k\|^2 \ge (\nabla f_{k+1} - \nabla f_k)^T \bm{d}_k \ge (\sigma-1) \nabla f_k^T \bm{d}_k$.
	This gives a lower bound on the step size: $\alpha_k \ge \frac{\sigma-1}{L} \frac{\nabla f_k^T \bm{d}_k}{\|\bm{d}_k\|^2}$.
	
	Now, sum the first Wolfe (Armijo) condition over all iterations:
	$$ f_{k+1} \le f_k + \rho \alpha_k \nabla f_k^T \bm{d}_k $$
	$$ f_{N+1} - f_0 = \sum_{k=0}^N (f_{k+1} - f_k) \le \rho \sum_{k=0}^N \alpha_k \nabla f_k^T \bm{d}_k $$
	Since $f$ is bounded below, as $N \to \infty$, the sum on the right must converge. Since $\rho>0$ and $\nabla f_k^T \bm{d}_k < 0$, the series $\sum_{k=0}^\infty -\alpha_k \nabla f_k^T \bm{d}_k$ must converge.
	Substituting the lower bound for $\alpha_k$:
	$$ \sum_{k=0}^{\infty} -\alpha_k \nabla f_k^T \bm{d}_k \ge \sum_{k=0}^{\infty} -\frac{1-\sigma}{L} \frac{(\nabla f_k^T \bm{d}_k)^2}{\|\bm{d}_k\|^2} $$
	By definition, $\cos \theta_k = \frac{-\nabla f_k^T \bm{d}_k}{\|\nabla f_k\| \|\bm{d}_k\|}$. So $(\nabla f_k^T \bm{d}_k)^2 = \cos^2 \theta_k \|\nabla f_k\|^2 \|\bm{d}_k\|^2$.
	$$ \sum_{k=0}^{\infty} \frac{1-\sigma}{L} \cos^2 \theta_k \|\nabla f_k\|^2 < \infty $$
	Since $(1-\sigma)/L$ is a positive constant, the result follows.
\end{proof}

% 'corollary' is defined via \newtheorem -> use optional title + \label
\begin{corollary}[Convergence Corollary]\label{cor:convergence}
	If an algorithm produces search directions such that the angle $\theta_k$ is bounded away from $90^\circ$ (i.e., $\cos \theta_k \ge \delta > 0$ for all $k$), then Zoutendijk's theorem implies that $\lim_{k \to \infty} \|\nabla f(\bm{x}_k)\| = 0$. This guarantees convergence to a stationary point. This condition on $\cos \theta_k$ is crucial and is satisfied by many algorithms, including the steepest descent and quasi-Newton methods.
\end{corollary}


\section{Trust Region Methods}

Trust region methods are a powerful class of algorithms for unconstrained optimization. Unlike line search methods, which select a direction and then a step length, trust region methods build a local model of the objective function and restrict the search for the next iterate to a region around the current point where the model is considered reliable.

\subsection{Basic Idea and Motivation}

The core idea is: at each iteration, construct a (usually quadratic) model $q_k(\bm{d})$ of $f$ near the current point $\bm{x}_k$, and only trust this model within a ball of radius $\Delta_k$ (the trust region). The next step $\bm{d}_k$ is chosen by (approximately) minimizing $q_k(\bm{d})$ subject to $\|\bm{d}\| \le \Delta_k$.

\begin{definition}{Trust Region Subproblem}{def:tr_subproblem}
    At iteration $k$, the trust region subproblem is:
    $$
    \min_{\bm{d} \in \mathbb{R}^n} \quad q_k(\bm{d}) = f(\bm{x}_k) + \nabla f(\bm{x}_k)^T \bm{d} + \frac{1}{2} \bm{d}^T B_k \bm{d}
    $$
    subject to $\|\bm{d}\| \le \Delta_k$, where $B_k$ is the Hessian $\nabla^2 f(\bm{x}_k)$ or its approximation.
\end{definition}

\subsection{Trust Region Models}

The model $q_k(\bm{d})$ is typically quadratic, capturing local curvature information. The choice of $B_k$ affects the algorithm's behavior:
\begin{itemize}
    \item If $B_k = \nabla^2 f(\bm{x}_k)$, the model is locally accurate (Newton-type).
    \item If $B_k$ is positive definite, the subproblem is easier to solve and ensures descent.
    \item If $B_k$ is an approximation (e.g., quasi-Newton), the method is more robust for large-scale problems.
\end{itemize}

\subsection{Solving the Trust Region Subproblem}

The trust region subproblem is a constrained quadratic minimization. There are several solution strategies:

\begin{itemize}
    \item \textbf{Cauchy Point}: Minimizes $q_k(\bm{d})$ along the steepest descent direction $-\nabla f(\bm{x}_k)$, clipped to the boundary of the trust region. Fast to compute, guarantees sufficient decrease.
    \item \textbf{Dogleg Method}: For positive definite $B_k$, combines the steepest descent direction and the Newton direction, choosing a step along a piecewise linear path (the "dogleg") within the trust region.
    \item \textbf{Exact Solution}: For small $n$, the subproblem can be solved exactly using eigenvalue decomposition or the Moré-Sorensen algorithm.
    \item \textbf{Truncated Conjugate Gradient}: For large-scale problems, an iterative method is used to approximately solve the subproblem.
\end{itemize}

\begin{definition}{Cauchy Point}{def:cauchy_point}
    The Cauchy point $\bm{d}_C$ is the minimizer of $q_k(\bm{d})$ along $-\nabla f(\bm{x}_k)$ within the trust region:
    $$
    \bm{d}_C = -\tau_k \nabla f(\bm{x}_k)
    $$
    where $\displaystyle\tau_k$ is chosen so that $\displaystyle\|\bm{d}_C\| = \min\left\{ \frac{\|\nabla f(\bm{x}_k)\|^2}{\nabla f(\bm{x}_k)^T B_k \nabla f(\bm{x}_k)}, \Delta_k \right\}$.
\end{definition}

\begin{definition}{Dogleg Method}{def:dogleg}
    The dogleg method constructs a path from the origin to the Cauchy point, then to the full Newton step. The step $\bm{d}_{DL}$ is chosen along this path such that $\|\bm{d}_{DL}\| \le \Delta_k$.
\end{definition}

\subsection{Trust Region Radius Update}

After computing $\bm{d}_k$, we evaluate how well the model predicted the actual reduction in $f$:

\begin{definition}{Trust Region Ratio}{def:trust_region_ratio}
    The ratio $\rho_k$ is defined as
    $$
    \rho_k = \frac{f(\bm{x}_k) - f(\bm{x}_k + \bm{d}_k)}{q_k(\bm{0}) - q_k(\bm{d}_k)}
    $$
    It measures the agreement between the model and the true function.
\end{definition}

The update strategy is:
\begin{itemize}
    \item If $\rho_k \ge \eta_1$ (e.g., $\eta_1 = 0.75$), the model is good: accept the step and increase $\Delta_{k+1}$.
    \item If $\eta_2 \le \rho_k < \eta_1$ (e.g., $\eta_2 = 0.1$), accept the step, keep $\Delta_{k+1}$ unchanged.
    \item If $\rho_k < \eta_2$, reject the step, decrease $\Delta_{k+1}$.
\end{itemize}

\subsection{Global Convergence}

Trust region methods can guarantee global convergence under mild assumptions:

\begin{theorem}{Global Convergence of Trust Region Methods}{theo:tr_global_convergence}
    If $f$ is bounded below and continuously differentiable, and the model $q_k$ satisfies regularity conditions, then any limit point of the sequence $\{\bm{x}_k\}$ generated by the trust region method is a stationary point of $f$.
\end{theorem}

\end{document}